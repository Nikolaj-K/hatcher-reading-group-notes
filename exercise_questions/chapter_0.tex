
\noindent{\bf{1.}} Construct an explicit deformation retraction of the torus with one point deleted onto a graph consisting of two circles intersecting in a point, namely, longitude and meridian circles of the torus.
\\
\\
\noindent{\bf{2.}} Construct an explicit deformation retraction of $\R^n-\{0\}$ onto $S^{n-1}$.
\\
\\
\noindent{\bf{3.}}
\begin{itemize}
\item[(a)] Show that the composition of homotopy equivalences $X\to Y$ and $Y\to Z$ is a homotopy equivalence $X\to Z$. Deduce that homotopy equivalence is an equivalence relation.
\item[(b)] Show that the relation of homotopy among maps $X\to Y$ is an equivalence relation.
\item[(c)] Show that a map homotopic to a homotopy equivalence is a homotopy equivalence.
\end{itemize}

\noindent{\bf{4.}} A {\bf{deformation retraction in the weak sense}} of a space $X$ to a subspace $A$ is a homotopy $f_t\colon X\to X$ such that $f_0=\1$, $f_1(X)\subset A$, and $f_t(A)\subset A$ for all $t$. Show that if $X$ deformation retracts to $A$ in this weak sense, then the inclusion $A\hookrightarrow X$ is a homotopy equivalence.
\\
\\
\noindent{\bf{5.}} Show that if a space $X$ deformation retracts toa  point $x\in X$, then for each neighborhood $U$ of $x$ in $X$ there exists a neighborhood $V\subset U$ of $x$ such that the inclusion map $V\hookrightarrow U$ is nullhomotopic.
\\
\\
\noindent{\bf{6.}}
\begin{itemize}
\item[(a)] Let $X$ be the subspace of $\R^2$ consisting of the horizontal segment $[0,1]\times\{0\}$ together with all the vertical segments $\{r\}\times [0,1-r]$ for $r$ a rational number in $[0,1]$. Show that $X$ deformation retracts to any point in the segment $[0,1]\times\{0\}$, but not to any other point. [See the preceding problem.]
\item[(b)] Let $Y$ be the subspace of $\R^2$ that is the union of an infinite number of copies of $X$ arranged as in the figure on Hatcher, pg 18. Show that $Y$ is contractible but does not deformation retract onto any point.
\item[(c)] Let $Z$ be the zigzag subspace of $Y$ homeomorphic to $\R$ indicated by the heavier line. Show there is a deformation retraction in the weak sense (see Exercise 4) of $Y$ onto $Z$, but no true deformation retraction.
\end{itemize}

\noindent{\bf{7.}} Fill in the details in the following construction from [Edwards 1999] of a compact space $Y\subset \R^3$ with the same properties as the space $Y$ in Exercise 6, that is, $Y$ is contractible but does not deformation retract to any point. To begin, Let $X$ be the union of an infinite sequence of cones on the Cantor set arranged end-to-end, as in the figure on Hatcher, pg 18. Next, form the one-point compactification of $X\times\R$. This embeds in $\R^3$ as a closed disk with curved `fins' attached along circular arcs, and with the one-point compactification of $X$ as a cross-sectional slice. The desired space $Y$ is then obtained from this subspace of $\R^3$ by wrapping one more cone on the Cantor set around the boundary of the disk.
\\
\\
\noindent{\bf{8.}} For $n>2$, construct an $n-$room analog of the house with two rooms.
\\
\\
\noindent{\bf{9.}} Show that a retract of the contractible space is contractible.
\\
\\
\noindent{\bf{10.}} Show that a space $X$ is contractible iff every map $f\colon X\to Y$, for arbitrary $Y$, is nullhomotopic. Similarly, show $X$ is contractible iff every map $f\colon Y\to X$ is nullhomotopic.
\\
\\
\noindent{\bf{11.}} Show that $f\colon  X\to Y$ is a homotopy equivalence if there exist maps $g,h\colon Y\to X$ such that $fg\simeq\1$ and $hf\simeq\1$. More generally, show that $f$ is a homotopy equivalence if $fg$ and $hf$ are homotopy equivalences.
\\
\\
\noindent{\bf{12.}} Show that a homotopy equivalence $f\colon X\to Y$ induces a bijection between the set of path-components of $X$ and the set of path-components of $Y$, and that $f$ restricts to a  homotopy equivalence from each path-component of $X$ to the corresponding path component of $Y$. Prove also the corresponding statemnts with components instead of path-components. Deduce that if the components of a space $X$ coincide with its path-components, then the same holds for any space $Y$ homotopy equivalent to $X$.
